\documentclass[12pt,a4paper]{amsart}
% ukazi za delo s slovenscino -- izberi kodiranje, ki ti ustreza
\usepackage[slovene]{babel}
\usepackage[utf8]{inputenc}
%\usepackage[T1]{fontenc}
\usepackage{amsmath,amssymb,amsfonts}
\usepackage{url}
%\usepackage[normalem]{ulem}
\usepackage[dvipsnames,usenames]{color}
\usepackage{caption}
\usepackage{lipsum}
\usepackage{tikz}

\usetikzlibrary{graphs}
\usetikzlibrary{graphs.standard}

%\makeatletter
%\renewcommand\section{\@startsection{section}{1}%
%  \z@{.5\linespacing\@plus.7\linespacing}{.5\linespacing}%
%  {\normalfont\scshape\large\centering}}
%\renewcommand\subsection{\@startsection{subsection}{2}%
%  \z@{.5\linespacing\@plus.7\linespacing}{.5\linespacing}%
%  {\normalfont\scshape}}
%\renewcommand\subsubsection{\@startsection{subsubsection}{3}%
%  \z@{.5\linespacing\@plus.7\linespacing}{-.5em}%
%  {\normalfont\itshape}}
%\makeatother

% ne spreminjaj podatkov, ki vplivajo na obliko strani
\textwidth 15cm
\textheight 24cm
\oddsidemargin.5cm
\evensidemargin.5cm
\topmargin-5mm
\addtolength{\footskip}{10pt}
\pagestyle{plain}
\overfullrule=15pt % oznaci predlogo vrstico


% ukazi za matematicna okolja
\theoremstyle{definition} % tekst napisan pokoncno
\newtheorem{definicija}{Definition}[section]
\newtheorem{primer}[definicija]{Example}
\newtheorem{opomba}[definicija]{Remark}

\renewcommand\endprimer{\hfill$\diamondsuit$}

\theoremstyle{plain} % tekst napisan posevno
\newtheorem{lema}[definicija]{Lemma}
\newtheorem{izrek}[definicija]{Theorem}
\newtheorem{trditev}[definicija]{Statement}
\newtheorem{posledica}[definicija]{Corollary}
\newtheorem{conjecture}[definicija]{Conjecture}


% za stevilske mnozice uporabi naslednje simbole
\newcommand{\R}{\mathbb R}
\newcommand{\N}{\mathbb N}
\newcommand{\Z}{\mathbb Z}
\newcommand{\C}{\mathbb C}
\newcommand{\Q}{\mathbb Q}

% ukaz za slovarsko geslo
\newlength{\odstavek}
\setlength{\odstavek}{\parindent}
\newcommand{\geslo}[2]{\noindent\textbf{#1}\hspace*{3mm}\hangindent=\parindent\hangafter=1 #2}

% naslednje ukaze ustrezno popravi
\newcommand{\program}{Finančna matematika} % ime studijskega programa: Matematika/Finančna matematika
\newcommand{\imeavtorja}{Marcel Blagotinšek, Peter Milivojević} % ime avtorja
\newcommand{\imementorja}{doc. dr. Janoš Vidali} % akademski naziv in ime mentorja
\newcommand{\imesomentorja}{prof. dr. Riste Škrekovski}
\newcommand{\naslovdela}{Maximum number of edges in a connected graph with n vertices and diameter d}
\newcommand{\letnica}{2023} %letnica diplome

\begin{document}

\thispagestyle{empty}
\noindent{\large
UNIVERZA V LJUBLJANI\\[1mm]
FAKULTETA ZA MATEMATIKO IN FIZIKO\\[5mm]
\program\ }
\vfill

\begin{center}{\large
\imeavtorja\\[2mm]
{\bf \naslovdela}\\[10mm]
Skupinski projekt\\[2mm]
Kratek opis problema\\[1cm]
Advisers: \imementorja, \\ \imesomentorja\\[2mm]}
\end{center}
\vfill

\noindent{\large
Ljubljana, \letnica}
\pagebreak

\section{Navodilo naloge}

A connected graph with diameter $d$ on $n$ vertices with the minimal number of edges will be a
tree and henceforth, it will have $n - 1$ edges. It will be harder to answer which graphs on a
fixed number of vertices $n$ and fixed diameter $d$ have the maximal number of edges. We want to
analyse the structure of such graphs. So, for a fixed number of vertices $n$ and a fixed diameter
$d$, when these two values are small, apply an exhaustive search. Next, for larger $n$ and $d$, apply
some metaheuristic. Try to obtain some specific properties of these graphs. Verify for how large
$n$ and $d$ your exhaustive search and your metaheuristic implementations are efficent.


\section{Opis problema}

Naloga nam zastavlja problem ugotovitve največjega možnega števila povezav v povezanih grafih
z določenim številom točk $n$ in določenim premerom $d$. Za $d = 1$ ugotovimo, da je ne glede
na izbiro števila vozlišč $n$, iskani graf ravno polni graf in ima posledično
$\frac{n (n - 1)}{2}$ povezav. V nasledjem koraku hitro ugotovimo, da se pri $d = 2$ število
povezav zmanjša le za $1$, saj se z odstranitvijo katere koli poljubne povezave v polnem grafu
premer poveča na $d = 2$ in ker smo za to potrebovali odstraniti le eno samo povezavo je največje
možno število povezav v grafu z $n$ točkami in premerom $d = 2$ enako $\frac{n (n - 1)}{2} - 1$.
Podobno opazimo, da so grafi za premere $d = n - 1$ ravno drevesa s stopnjo $2$ in je zato število
povezav enako $n - 1$. Tako nas pri dani nalogi v resnici zanimajo predvsem grafi za katere velja
$d \in \{3, \dots, n - 2\}$.


\section{Potek dela}

Nalogo sva pričela reševati z opazovanjem in računanjem grafov z manjšim številom vozlišč $n$, pri tem sva si
pomagala tudi z algoritmom napisanim spodaj. Opazila sva, da so iskani grafi za premer $d = 1$ polni grafi in
imajo kot taki $\frac{n (n - 1)}{2}$ povezav. Za premer $d = 2$ sva opazila, da je potrebno odstraniti
polnemu grafu le eno povezavo in je tako maksimalno število povezav enako $\frac{n (n - 1)}{2} - 1$. Grafi s
premerom $d = n - 1$ so prav tako enolično določeni kot drevesa s stopnjo 2 in imajo tako $n - 1$ povezav.
Tako sva nadaljevala z reševanjem jedra problema, ki so grafi s premerom $d \in \{3, \dots, n - 2\}$.\\

\begin{verbatim}
    def find_connected_graph_with_diameter(n, d):
        max_edges = 0
        max_edges_graph = None
        for G in graphs.nauty_geng(str(n) + " -c"):
            diameter = G.diameter()
            if diameter == d:
                num_edges = G.size()
                if num_edges > max_edges:
                    max_edges = num_edges
                    max_edges_graph = G.copy()
    return max_edges_graph, max_edges

import pandas as pd
import matplotlib.pyplot as plt

results = []
for n in range(1, 10):
    for d in range(1, n):
        max_edges_graph, max_edges = find_connected_graph_with_diameter(n, d)
        result_dict = {
            'n': n,
            'd': d,
            'max_edges': max_edges
        }
        results.append(result_dict)

df = pd.DataFrame(results)
print(df)

plt.figure(figsize=(10, 6))
for n in range(1, 9):
    subset = df[df['n'] == n]
    plt.plot(subset['d'], subset['max_edges'], label=f'n={n}')
plt.xlabel('diameter (d)')
plt.ylabel('maximum Edges')
plt.legend()
plt.title('max_edges(d) for different n')
plt.show()

\end{verbatim}

Napisani algoritem je dal podatke o največjem možnem številu povezav za grafe do $10$ vozlišč,
ki sva jih uredila v sledečo tabelo:

\begin{center}
    \begin{tabular}{||c c c c c c c c c c||} 
     \hline
     n\textbackslash d & 1 & 2 & 3 & 4 & 5 & 6 & 7 & 8 & 9 \\ [0.5ex] 
     \hline\hline
     2 & 1 & & & & & & & & \\ 
     \hline
     3 & 3 & 2 & & & & & & & \\ 
     \hline
     4 & 6 & 5 & 3 & & & & & & \\
     \hline
     5 & 10 & 9 & 6 & 4 & & & & & \\
     \hline
     6 & 15 & 14 & 10 & 7 & 5 & & & & \\ 
     \hline
     7 & 21 & 20 & 15 & 11 & 8 & 6 & & & \\ 
     \hline
     8 & 28 & 27 & 21 & 16 & 12 & 9 & 7 & & \\ 
     \hline
     9 & 36 & 35 & 28 & 22 & 17 & 13 & 10 & 8 & \\ 
     \hline
     10 & 45 & 44 & 36 & 29 & 23 & 18 & 14 & 11 & 9 \\ 
     \hline
    \end{tabular}
\end{center}

% tle bo treba še elegantno dodat tvoj postopek izračuna
Iz tabele smo s sledečim računom prišli do formule, ki nam za $d > 1$ pove maksimalno število povezav za
grafe do $n = 10$ vozlišč in morda še več, za kar trenutno ne moremo še zagotovo trditi.\\


Z opazovanjem generiranih grafov sva opazila, da vsi grafi vsebujejo poln podgraf velikosti $n - d + 1$.
V nadaljevanju smo opazili, da imajo grafi za premere $d > 1$ poleg $\frac{(n - d + 1)(n - d)}{2}$ povezav (zaradi
prej opaženega polnega grafa velikosti $n - d + 1$) dodatno še $n - 2$ povezav, ki niso enolično določene.
Tako imamo ponovno podano enako formulo ki nam za grafe s premerom $d > 2$ pove, da je maksimalno število
povezav enako $\frac{(n - d + 1)(n - d)}{2} + n - 2$.\\


V drugem delu naloge se bova problema lotila s metahurističnim pristopom.


\end{document}
