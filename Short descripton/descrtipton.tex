\documentclass[12pt,a4paper]{amsart}
% ukazi za delo s slovenscino -- izberi kodiranje, ki ti ustreza
\usepackage[slovene]{babel}
\usepackage[utf8]{inputenc}
%\usepackage[T1]{fontenc}
\usepackage{amsmath,amssymb,amsfonts}
\usepackage{url}
%\usepackage[normalem]{ulem}
\usepackage[dvipsnames,usenames]{color}
\usepackage{caption}
\usepackage{lipsum}
\usepackage{tikz}

\usetikzlibrary{graphs}
\usetikzlibrary{graphs.standard}

%\makeatletter
%\renewcommand\section{\@startsection{section}{1}%
%  \z@{.5\linespacing\@plus.7\linespacing}{.5\linespacing}%
%  {\normalfont\scshape\large\centering}}
%\renewcommand\subsection{\@startsection{subsection}{2}%
%  \z@{.5\linespacing\@plus.7\linespacing}{.5\linespacing}%
%  {\normalfont\scshape}}
%\renewcommand\subsubsection{\@startsection{subsubsection}{3}%
%  \z@{.5\linespacing\@plus.7\linespacing}{-.5em}%
%  {\normalfont\itshape}}
%\makeatother

% ne spreminjaj podatkov, ki vplivajo na obliko strani
\textwidth 15cm
\textheight 24cm
\oddsidemargin.5cm
\evensidemargin.5cm
\topmargin-5mm
\addtolength{\footskip}{10pt}
\pagestyle{plain}
\overfullrule=15pt % oznaci predlogo vrstico


% ukazi za matematicna okolja
\theoremstyle{definition} % tekst napisan pokoncno
\newtheorem{definicija}{Definition}[section]
\newtheorem{primer}[definicija]{Example}
\newtheorem{opomba}[definicija]{Remark}

\renewcommand\endprimer{\hfill$\diamondsuit$}

\theoremstyle{plain} % tekst napisan posevno
\newtheorem{lema}[definicija]{Lemma}
\newtheorem{izrek}[definicija]{Theorem}
\newtheorem{trditev}[definicija]{Statement}
\newtheorem{posledica}[definicija]{Corollary}
\newtheorem{conjecture}[definicija]{Conjecture}


% za stevilske mnozice uporabi naslednje simbole
\newcommand{\R}{\mathbb R}
\newcommand{\N}{\mathbb N}
\newcommand{\Z}{\mathbb Z}
\newcommand{\C}{\mathbb C}
\newcommand{\Q}{\mathbb Q}

% ukaz za slovarsko geslo
\newlength{\odstavek}
\setlength{\odstavek}{\parindent}
\newcommand{\geslo}[2]{\noindent\textbf{#1}\hspace*{3mm}\hangindent=\parindent\hangafter=1 #2}

% naslednje ukaze ustrezno popravi
\newcommand{\program}{Finančna matematika} % ime studijskega programa: Matematika/Finančna matematika
\newcommand{\imeavtorja}{Marcel Blagotinšek, Peter Milivojević} % ime avtorja
\newcommand{\imementorja}{doc. dr. Janoš Vidali} % akademski naziv in ime mentorja
\newcommand{\imesomentorja}{prof. dr. Riste Škrekovski}
\newcommand{\naslovdela}{Maximum number of edges in a connected graph with n vertices and diameter d}
\newcommand{\letnica}{2023} %letnica diplome

\begin{document}

\thispagestyle{empty}
\noindent{\large
UNIVERZA V LJUBLJANI\\[1mm]
FAKULTETA ZA MATEMATIKO IN FIZIKO\\[5mm]
\program\ }
\vfill

\begin{center}{\large
\imeavtorja\\[2mm]
{\bf \naslovdela}\\[10mm]
Skupinski projekt\\[2mm]
Kratek opis\\[1cm]
Advisers: \imementorja, \\ \imesomentorja\\[2mm]}
\end{center}
\vfill

\noindent{\large
Ljubljana, \letnica}
\pagebreak

\section{Navodilo naloge}

A connected graph with diameter $d$ on $n$ vertices with the minimal number of edges will be a
tree and henceforth, it will have $n - 1$ edges. It will be harder to answer which graphs on a
fixed number of vertices $n$ and fixed diameter $d$ have the maximal number of edges. We want to
analyse the structure of such graphs. So, for a fixed number of vertices $n$ and a fixed diameter
$d$, when these two values are small, apply an exhaustive search. Next, for larger $n$ and $d$, apply
some metaheuristic. Try to obtain some specific properties of these graphs. Verify for how large
$n$ and $d$ your exhaustive search and your metaheuristic implementations are efficent.


\section{Opis problema}

Želiva poiskati povezane grafe na $n$ vozliščih s premerom $d$, ki bodo imeli maksimalno število povezav. Najin cilj je, na podlagi 
testiranja oz.\ generiranja, pridobiti kar se da dober vpogled v strukturo teh grafov in posledično ugotoviti, če za njih veljajo kakšne 
posebne lastnosti. Za majhne vrednosti $n$ in $d$, se bova problema lotila z generiranjem grafov, za večje pa bova uporabila metodo simulated 
annealing. Ugotavljala bova tudi efektivnost najinih metod v odvisnosti od vrednosti $n$ in $d$.


\section{Potek Dela}

Ideja prve faze projekta t.i.\ exhaustive search-a je, da z generiranjem vseh možnih povezanih grafov na $n$ vozliščih s premerom $d$, 
poiščeva tiste, ki imajo maksimalno število povezav. To bova počela za majhne vrednosti $n$ in $d$. Kako majhne, bo odvisno od časovne 
zahtevnosti samega algoritma, kajti je pričakovati, da bo že pri ne malo od 5 večjih vrednostih $n$ algoritem počasen. Na podlagi 
generiranja grafov za različne $n$ in $d$ bova poskušala ugotoviti kakšne lastnosti, tako strukturne kot vizualne, lahko pripiševa 
tem grafom. Naraščanje/padanje števila povezav v odvisnosti od števila vozlišč oz.\ premetra bova prikazala tudi s pomočjo grafa, 
ki se bo morda obnašal podobno kot kakšna znana funkcija, kar bo vsekakor pomagalo pri oceni števila povezav za večje vrednosti $n$ 
in $d$. Kot omenjeno bova poskusila najti kakšno formulo za maksimalno število povezav pri številu vozlišč $n$ in premeru $d$. Tako 
pridobljene formule, četudi bo morda držala, ne bova dokazovala in jo bova posledično uporabila kot oceno v primeru generiranja grafov. 
Na koncu bova poleg ugotovitev glede lastnosti grafov v poročilu zapisala tudi pri kako velikih vrednostih $n$ in $d$ je najin algoritem 
prenehal učinkovito delovati. V drugi fazi projekta se bova problema lotila z metahevristično metodo simulated annealing. Začela bova z 
nekim začetnim povezanim grafom $G$, ki bo ustrezal pogojem $n$ in $d$, nato pa bova dodala povezavo iz množice povezav komplementa grafa $G$. 
V kolikor bo premer grafa $G + e$ ostal isti, imamo nov graf, ki ima isti premer vendar povezavo več. Če bo premer novega grafa manjši od $d$, 
pa bova poiskala vozlišči $u$ in $v$ na maksimalni razdalji in odstranjevala povezave iz poti med $u$ in $v$ toliko časa, dokler ne bo premer 
spet $d$. Seveda se lahko zgodi, da bo premer večji od $d$, takrat pa bova spet poiskala vozlišči $u$ in $v$ na maksimalni razdalji in dodajala 
neke povezave na poti med $u$ in $v$ toliko časa, dokler ne bo premer spet $d$. Povezave bova morala dodajati med ustreznimi vozlišči. Torej, 
če bo nov premer $d-1$, bova dodala povezavo med vozliščema na oddaljenosti 2. Pri tem se zavedava, da z neko verjetnostjo v nekem koraku vzameva 
graf z manj povezavami, ki pa je morda boljše izhodišče za naprej. \pagebreak Tudi tukaj bova začela na manjših vrednostih, in s tem preveriva, če najin 
algoritem deluje, nato pa $n$ in $d$ povečujeva. Tudi v drugi fazi projekta bova pozorna na efektivnost oz. časovno zahtevnost, ter bova ugotovitve 
glede tega zapisala v poročilu.\\
Algoritme in programe bova v obeh fazah pisala v CoCalc Jupyter notebook-u.

\end{document}
